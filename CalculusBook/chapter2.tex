% \chapterimage{ch2Head.png}

\chapter{Discovering what to do...}

\section{First ideas}\index{First ideas}
So, now here you have your first astronomy picture, \footnote{For example purposes the image selected is a picture of M83 through a Wide H-alpha and [N II] filter. } what do you see?, it is a monochrome image, with different levels of brightness, slightly big (8500 x 5000), it looks like a lot of stars making a spiral.

\subsection{PCA}

Welcome to Astronomy where you will find more acronyms than words to mention something on articles, lots of fun!, well in this case PCA stands for Principal Component Analysis, the objective of this method is to reduce dimnensionality, tranform the data to another space where is can be manipulated and reduced, there are multiple examples of work that has been done in astronomy applying this technique.

Therefor, the idea of applying this method is that if we have multiple-wavelenght images of the same target and transform them to PCA space then we will have less dimensionality and it will be easier to process all the data and fins valuable information.\footnote{Before I forget to mention, later I discovered that PCA is not comonly used for datamining preprocessing because it is hard to interpret the information in the output result. Imagine clusters of data on PCA space, how do you make sense to that?}


Any questions you may have and how to install, here is my GitHub page for software tools \url{https://github.com/LaurethTeX/Clustering/blob/master/Tools.md} 