\chapterimage{ch1Head.jpg} % Chapter heading image

\chapter{函数、极限与连续性}

\section{预备知识}\index{yubeizhishi}

\subsection{不等式}\index{inequality}

\begin{enumerate}
\item Bernoulli不等式:$a_i>1, i = 1,2,\cdots, n$且$\forall i, a_i$同号,就有
\begin{equation}\label{bernoulli}
  \prod_{i=1}^{n}(1+a_i) \geq 1+\sum_{i=1}^{n}a_i
\end{equation}
例:$x_n=(1+\frac{1}{n})^n$,令$a_i=\frac{1}{n}$,带入\ref{bernoulli}中,可得$2\leq x_n \leq 3$
\item 
\end{enumerate}



This means that from multiple images with different wavelengths of the same target apply an algorithm to find the hidden patterns that lie hidden between them.

\section{A bit of context}\index{Context}
Ok, here is where I explain from where this is going to start, at that time I just had a microcontrollers and engineering design course my mind was set completelly to find appplicable theories and create uselful things with them, which is the complete opposite of how astronomy works. First, there's no way to test an experiment with galaxies and most of the information is fuzzy and subjective (not all). The process of having an, let's say \emph{astronomy idea} is a result of applying all your physics knowledge and consider the \textbf{cosmological principle},
\begin{quote}
The (testable) assumption that the same physical laws that apply here and now also apply everywhere and at all times, and that there are no special locations or directions in the universe.
\end{quote}

That's how science is made, thinking and testing and thinking again, creating your own scientific method, comming up with hypothesis, learning what might work and what not, using your insticts.

Well, before comming here I didn't think like that, it was just all about being super productive and thinking about doing robots and all kinds of devices with sensors. I had some experience programming in C/C++, no computer science backgound and I had never had an astronomy course.

This report was written in order to help someone to continue researching about data mining techniques applied in Astronomy, I explain how did I come up with the clustering techniques, my hypothesis, some tests and other ideas I have had, I hope this can help anyone and the research is continued. Anything you may need/questions do not hesitate to contact me, my e-mail address is: \emph{mrs.petzl@gmail.com}, also s part of my own documentation I created a GitHub page where you can download all the codes I programmed and find more information. The link to this page is: \url{https://github.com/LaurethTeX/Clustering}, from the \textsc{readme} file you can acces to all the pages, take your time to surf.
%------------------------------------------------