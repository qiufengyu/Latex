% \chapterimage{ch3Head}

\chapter{Understand your data}

Before continuing, first and most importantly you must select the \emph{raw} data you are going to process and later after you aquire experience with an specific dataset the idea is to expand the algorithms to any kind of dataset. The important things are to learn how to input the data correctly, establish the right \emph{learning paramenters} in the selected algorithm and find the best way to visualize your results and interpret them correctly.

Now let's start with basic concepts that vary from an engineering to an astronomer point of view.
\section{What is an image?}
	As you may know, an image is a matrix of numbers that cointains the specific brightness level that corresponds to a given pixel. And from there the concepts evolves and adds channels of colour and depth. But for now, let's just think about monochromatic images (only one channel). In Astronomy, images are usually considered sets of scientific data, observations, that contain information about an speficic target in the sky seen throught an specific filter and the levels of brightness correspond to the behaviour of the optical sensor (CCD camera) in relation with the number of electrons that hit a particular pixel through an specific waveband. Something else to consider is that the sky is not flat with this I mean that the celestial vault is like a sphere surrounding us therefore cartesian coorditanes are not the paramenters used to identify points in space, there is another system called WCS (World Coordinate System) hence a conversion between pixels and WCS coordinates exists. As you are realizing now just one image can contain tons of information related to it, now imagine that multiplied for terabytes and terabytes of stars, galaxies, planets, nebulae or any object in space. Fortunately in astronomy this is solved using an image format that cointains the image and its own information.
	\subsection{FITS files}
    	This format is the standard data format used in astronomy, can contain one image, multiple images, tables and header keywords providing descriptive information about the data. The way it works is that this format can contain a text file with keywords that comprise the information about the observation and a multidimensional array that could be a table, or an image, or an array of images (data cube). This files can be managed in diffetent ways, with an image preview use DS9, for handing the data in a program use the \emph{Python} package \emph{PyFITS}.

	\subsection{WFC3 ERS M83 Data Products}
    The selected dataset to test the data mining libraries I found is a series of observations of M83 at 9 different wavelengths, the original images can be found in this webpage, the specific information about them can be found in Table. This particular images were observed through HST with the WFC3/UVIS camera.


%Los links que estan en la pagina de el curso de caltech
%Sky surveys