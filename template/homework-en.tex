\documentclass[11pt]{article}
\usepackage{amsmath,amssymb,amsthm}
\usepackage[top=2.54cm,bottom=2.54cm,left=3.18cm,right=3.18cm]{geometry}

\usepackage[tt=false]{libertine}

% \usepackage[slantfont,boldfont]{xeCJK}
% \setmainfont{TeX Gyre Pagella}
% \setmonofont{Monaco}
% \setsansfont{Trebuchet MS}

\setlength{\parindent}{0pt}
\setlength{\parskip}{5pt plus 1pt}
\setlength{\headheight}{13.6pt}
\newcommand\question[2]{\vspace{.25in}\hrule\textbf{#1. #2}\vspace{.5em}\hrule\vspace{.10in}}
\renewcommand\part[1]{\vspace{.10in}\textbf{(#1)  }}


\begin{document}
%Section A==============Change the values below to match your information==================
\newcommand\NAME{Godfray}         % your name
\newcommand\STUID{}               % your student id
\newcommand\HWNUM{0}              % the homework number

%Section B==============Put your answers to the questions below here=======================


\title{\bfseries \sffamily Course Title Assignment}
\author{\sffamily \NAME}

\maketitle

\question{1}{Gradients and Hessians}
Recall that a matrix $A\in \mathbb{R}^{n\times n}$ is symmetric if $A^T = A$, that is, $A_{ij} = A_{ji}$ for all $i$, $j$. Also recall the gradient $\nabla f(x)$ of a function $f: \mathbb{R}^n \rightarrow \mathbb{R}$, which is the $n$-vector of partial derivatives $$\nabla f(x) = \left[
                                         \begin{array}{c}
                                           \frac{\partial}{\partial x_1}f(x) \\
                                           \vdots \\
                                           \frac{\partial}{\partial x_n}f(x) \\
                                         \end{array}
                                       \right]
\text{ where } x = \left[
                    \begin{array}{c}
                     x_1 \\
                     \vdots \\
                     x_n
                   \end{array}
                   \right].$$

The hessian $\nabla^2f(x)$ of a function $f: \mathbb{R}^n \rightarrow \mathbb{R}$ is the $n\times n$ symmetric matrix of twice partial derivatives,$$\left[
                                 \begin{array}{cccc}
                                   \frac{\partial^2}{\partial x_1^2}f(x) & \frac{\partial^2}{\partial x_1\partial x_2}f(x) & \cdots & \frac{\partial^2}{\partial x_1\partial x_n}f(x) \\
                                   \frac{\partial^2}{\partial x_2\partial x_1}f(x) & \frac{\partial^2}{\partial x_2^2}f(x) & \cdots & \frac{\partial^2}{\partial x_2\partial x_n}f(x) \\
                                   \vdots & \vdots & \ddots & \vdots \\
                                   \frac{\partial^2}{\partial x_n \partial x_1}f(x) & \frac{\partial^2}{\partial x_n \partial x_2}f(x) & \cdots & \frac{\partial^2}{\partial x_n^2}f(x) \\
                                 \end{array}
                               \right].$$

\part{a}Let $f(x)=\frac{1}{2}x^TAx + b^Tx$, where $A$ is a symmetric matrix and $b\in \mathbb{R}^n$ is a vector. What is $\nabla f(x)$?

We have $\nabla f(x) = \left[
             \begin{array}{c}
               \frac{\partial}{\partial x_1}\frac{1}{2}x^TAx \\
               \vdots \\
               \frac{\partial}{\partial x_n}\frac{1}{2}x^TAx \\
             \end{array}
           \right] + \left[
             \begin{array}{c}
               \frac{\partial}{\partial x_1}b^Tx \\
               \vdots \\
               \frac{\partial}{\partial x_n}b^Tx \\
             \end{array}
           \right]$, considering $$\frac{\partial x^TAx}{\partial x_i} = \frac{\partial}{\partial x_i}\sum_{j=1}^{n}\sum_{k=1}^{n}x_ja_{jk}x_{k} = \frac{\partial}{\partial x_i}\left(\sum_{j\neq i}^{n}\sum_{k\neq i}^{n}x_ja_{jk}x_{k}+ \sum_{k \neq i}^{n}x_ia_{ik}x_k+\sum_{j \neq i}^{n} x_ja_{ji}a_i+x_ia_{ii}x_i\right) $$
           $$ = 0 + \sum_{k \neq i}^{n}a_{ik}x_k + \sum_{j \neq i}^{n} x_ja_{ji} + 2a_{ii}x_i= \sum_{k = 1}^{n}a_{ik}x_k + \sum_{j=1}^{n} x_ja_{ji} = A_{i*}^Tx+A_{*i}^Tx = \left((A+A^T)x\right)_{i}$$
that is, the i-th element of $\frac{\partial x^TAx}{\partial x_i}$ is exact inner product of i-th row vector of $A$ and $x$ plus i-th column vector of $A$ and $x$ (also i-th row of $A^T$ and $x$), and we get the vectorized representation. For a symmetric matrix $A$, we have $A^T=A$, and $\frac{\partial}{\partial x_i}\frac{1}{2}x^TAx = \frac{1}{2}(A^T+A)x=Ax$. Likewise and more easier,
$$\frac{\partial b^Tx}{\partial x_i} = \frac{\partial}{\partial x_i}\sum_{j=1}^{n}b_jx_j = b_i.$$
Finally, $\nabla f(x) = Ax+b$.

\part{b}Let $f(x)=g(h(x))$, where $g: \mathbb{R} \rightarrow \mathbb{R}$ is differentiable and $h: \mathbb{R}^n \rightarrow \mathbb{R}$ is differentiable. What is $\nabla f(x)$?

By rule of chain of derivatives, $$\nabla f(x) = \frac{\partial}{\partial x_i} g(h(x)) = \frac{\text{d}g(h(x))}{\text{d}h(x)}\times \frac{\partial h(x)}{\partial x_i} = g^\prime(h(x))\times \nabla h(x).$$

\part{c}Let $f(x)=\frac{1}{2}x^TAx + b^Tx$, where $A$ is symmetric and $b\in \mathbb{R}^n$ is a vector. What is $\nabla^2 f(x)$?

From 1(a), $\frac{\partial}{\partial x_i}\left(\frac{1}{2}x^TAx + b^Tx\right) = Ax+b$, and $$\frac{\partial^2}{\partial x_i\partial x_j}\left(\frac{1}{2}x^TAx + b^Tx\right) = \frac{\partial}{\partial x_j}(Ax+b)= \frac{\partial Ax}{\partial x_j} = \frac{\partial}{\partial x_j}\sum_{k=1}^{n}a_{ik}x_k = a_{ij}.$$
Still, A is symmetric, and $\frac{\partial^2f(x)}{\partial x_i\partial x_j} = A_{ij}$, $\nabla^2 f(x) = A$.

\part{d}Let $f(x)=g(a^Tx)$, where $g: \mathbb{R} \rightarrow \mathbb{R}$ is continuously differentiable and $a\in \mathbb{R}^n$ is a vector. What are $\nabla f(x)$ and $\nabla^2 f(x)$? (\textit{Hint: your expression for $\nabla^2 f(x)$ may have as few as 11 symbols, including $~^\prime$ and parentheses.})

By rule of chain of derivatives and 1(a), $$\frac{\partial}{\partial x_i} g(a^Tx) = \frac{\text{d}g(a^Tx)}{\text{d}(a^Tx)}\times \frac{\partial a^Tx}{\partial x_i} = g^{\prime}(a^Tx)a_i$$
and $\nabla f(x) = g^{\prime}(a^Tx)a;$
$$\frac{\partial^2}{\partial x_i \partial x_j} g(a^Tx) = \frac{\partial}{\partial x_j}g^{\prime}(a^Tx)a_i = \frac{\text{d}g^\prime (a^Tx)}{\text{d}a^Tx}\times \frac{\partial a^Tx}{\partial x_j} = g^{\prime\prime}(a^Tx)a_ia_j$$
and $\nabla^2 f(x) = g^{\prime\prime}(a^Tx)aa^T.$


\end{document}
