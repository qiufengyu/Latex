\documentclass[12pt]{article}
\usepackage[top=2.54cm,bottom=2.54cm,left=3.18cm,right=3.18cm]{geometry}
\usepackage[BoldFont,SlantFont,CJKchecksingle]{xeCJK}
\usepackage{amssymb,amsmath}

\parindent=2em
\linespread{1.5}
\setmainfont{Times New Roman}
\setCJKmainfont[BoldFont=SimHei]{SimSun}
\setCJKmonofont{SimSun}
\setCJKsansfont{KaiTi}

\renewcommand{\today}{\number\year 年 \number\month 月 \number\day 日}

\begin{document}

\title{\textbf{1. 疯狂年代}}
\author{《三体》 \quad \textsf{刘慈欣}}
\maketitle

中国,1967年。

“红色联合”对“四·二八兵团”总部大楼的攻击已持续了两天,他们的旗帜在大楼周围躁动地飘扬着,仿佛渴望干柴的火种。

“红色联合”的指挥官心急如焚,他并不惧怕大楼的守卫者,那二百多名“四·二八”战士,与诞生于l966年初、经历过大检阅和大串联的“红色联合”相比要稚嫩许多。他怕的是大楼中那十几个大铁炉子,里面塞满了烈性炸药,用电雷管串联起来,他看不到它们,但能感觉到它们磁石般的存在,开关一合,玉石俱焚,而“四·二八”的那些小红卫兵们是有这个精神力量的。比起已经在风雨中成熟了许多的第一代红卫兵,新生的造反派们像火炭上的狼群,除了疯狂还是疯狂。

大楼顶上出现了一个娇小的身影,那个美丽的女孩子挥动着一面“四·二八”的大旗,她的出现立刻招来了一阵杂乱的枪声,射击的武器五花八门,有陈旧的美式卡宾枪、捷克式机枪和三八大盖,也有崭新的制式步枪和冲锋枪——后者是在“八月社论”发表之后从军队中偷抢来的(注:1967年8月《红旗》杂志发表“揪军内一小撮”的社论,使冲击军区、抢夺军队枪支弹药的事件愈演愈烈,全国范围的武斗也进入高潮。)——连同那些梭标和大刀等冷兵器,构成了一部浓缩的近现代史……“四·二八”的人在前面多次玩过这个游戏,在楼顶上站出来的人,除了挥舞旗帜外,有时还用喇叭筒喊口号或向下撒传单,每次他们都能在弹雨中全身而退,为自己挣到了崇高的荣誉。这次出来的女孩儿显然也相信自己还有那样的幸运她挥舞着战旗,挥动着自己燃烧的青春,敌人将在这火焰中化为灰烬,理想世界明天就会在她那沸腾的热血中诞生……她陶醉在这鲜红灿烂的梦幻中,直到被一颗步枪子弹洞穿了胸膛,十五岁少女的胸膛是那么柔嫩,那颗子弹穿过后基本上没有减速,在她身后的空中发出一声啾鸣。年轻的红卫兵同她的旗帜一起从楼顶落下,她那轻盈的身体落得甚至比旗帜还慢,仿佛小鸟眷恋着天空。

红色联合的战士们欢呼起来,几个人冲到楼下,掀开四·二八的旗帜,抬起下面纤小的遗体,作为一个战利品炫耀地举了一段,然后将她高高地扔向大院的铁门,铁门上带尖的金属栅条大部分在武斗初期就被抽走当梭标了,剩下的两条正好挂住了她,那一瞬间,生命似乎又回到了那个柔软的躯体。红色联合的红卫兵们退后一段距离,将那个挂在高处的躯体当靶子练习射击,密集的子弹对她来说已柔和如雨,不再带来任何感觉,她那春藤般的手臂不时轻挥一下,仿佛拂去落在身上的雨滴,直到那颗年轻的头颅被打掉了一半,仅剩的一只美丽的眼睛仍然凝视着一九六七年的蓝天,目光中没有痛苦,只有凝固的激情和渴望。

其实,比起另外一些人来,她还是幸运的,至少是在为理想献身的壮丽激情中死去。这样的热点遍布整座城市,像无数并行运算的CPU,将“文革大革命”联为一个整体。疯狂如同无形的洪水,将城市淹没其中,并渗透到每一个细微的角落和缝隙。

在城市边缘的那所著名大学的操场上,一场几千人参加的批斗会已经进行了近两个小时。在这个派别林立的年代,任何一处都有错综复杂的对立派别在格斗。在校园中,红卫兵、文革工作组、工宣队和军宣队,相互之间都在爆发尖锐的冲突,而每种派别的内部又时时分化出新的对立派系,捍卫着各自不同的背景和纲领,爆发更为残酷的较量。但这次被批斗的反动学术权威,却是任何一方均无异议的斗争目标,他们也只能同时承受来自各方的残酷打击。

\end{document} 